%%%%%%%%%%%%%%%%%%%%%%%%%%%%%%%%%%%%%%%%%%%%%%%%%%%%%%%%%%%%%%%%%%
% NRSC/ISRO Beamer Presentation Showcase
% Version: 12.1
% Date: August 26, 2025
%Author : Carbform (carbform.github.io)
% Description: A showcase and template for the NRSCISRO.cls class.
%  This file demonstrates all standard features.
%
% Compiler: pdfLaTeX
%%%%%%%%%%%%%%%%%%%%%%%%%%%%%%%%%%%%%%%%%%%%%%%%%%%%%%%%%%%%%%%%%%
% --------------------------------------------------------------------
% Choose your font theme: 'sans' (modern, default) or 'serif' (classic)
% --------------------------------------------------------------------
\documentclass[serif]{NRSCISRO}

% --- BIBLATEX SETUP FOR CITATIONS (Optional) ---
% For this to work, you need a 'bibfile.bib' in your folder.
\usepackage[backend=biber, style=numeric]{biblatex}
\addbibresource{bibfile.bib}

% ---------------------------------------------
% --- Presentation Information ---
\title{ISRO/NRSC Beamer Theme Showcase}
\subtitle{A Full-Featured Presentation Template}
\author{Carbform}
\department{Sci/Er-SC, Water Resources Group}
\institute{National Remote Sensing Centre, ISRO}
\email{xxxxxx@nrsc.gov.in}
\date{\today}

% --- Set the path to YOUR footer logo ---
% This command is required for the small logo to appear in the footer.
\setfooterlogo{images/nrsc_small.png}

\begin{document}

% --- Title Page ---
% The [plain] option removes the header and footer for a clean look.
% --- MODIFIED FRAME BELOW ---
\begin{frame}[plain]
    % Added \centering to align all content, including the date
    \centering
    % Added the figure environment for the main logo
    \begin{figure}
        \includegraphics[height=2.15cm]{images/nrsc_small.png}
    \end{figure}
    \vfill 
    \maketitle
\end{frame}
% --- END OF MODIFICATION ---


% --- Outline ---
\begin{frame}{Outline}
    \begin{multicols}{2}
        \tableofcontents
    \end{multicols}
\end{frame}

% ===================================================================
\section{Introduction}
% ===================================================================

\begin{frame}{About This Theme}
    \begin{itemize}
        \item This theme provides a robust and standardized look for all NRSC/ISRO presentations.
        \item You can choose a font theme using class options:
        \begin{itemize}
            \item `documentclass[sans]{NRSCISRO}` for a modern, clean look.
            \item `documentclass[serif]{NRSCISRO}` for a classic, academic appearance.
        \end{itemize}
        \item The footer logo is set in the preamble with setfooterlogo command.
    \end{itemize}
\end{frame}

% ===================================================================
\section{Content Elements}
% ===================================================================

\begin{frame}{Standard Block Environments}
    \begin{block}{Standard Block}
        This is the default block environment, ideal for general content.
    \end{block}
    
    \begin{alertblock}{Alert Block}
        This block is for critical warnings and important information.
    \end{alertblock}

    \begin{exampleblock}{Example Block}
        This block is used for examples, code, or positive outcomes.
    \end{exampleblock}
\end{frame}

\begin{frame}{Custom Box Commands}
    These commands are for simple, framed text without a title bar.
    
    \infobox{This is an info box. Good for neutral or supplementary information.}
    \successbox{This is a success box. Ideal for positive results or key takeaways.}
    \alertbox{This is an alert box. Use it for critical warnings or important notes.}
\end{frame}

\begin{frame}{Figures and Columns}
    \begin{columns}[T] % The [T] option aligns the tops of the columns
        \begin{column}{0.5\textwidth}
            \begin{figure}
                \includegraphics[width=\textwidth]{images/black.png}
                \caption{A figure in the left column.}
                \source{ISRO Image Gallery}
            \end{figure}
        \end{column}
        \begin{column}{0.5\textwidth}
            \textbf{Multi-column layouts} are useful for placing text next to figures.
            \begin{itemize}
                \item This is the right column.
                \item It can contain text or lists.
                \item The `source{}` command adds a formatted source line below a figure.
            \end{itemize}
        \end{column}
    \end{columns}
\end{frame}

\begin{frame}{Citations}
    You can cite sources in your text, which will be numbered automatically \cite{Einstein1926}.
    
    \bigskip % A little space

    The full, formatted reference list will appear on the "References" slide at the end of the presentation.
\end{frame}

% ===================================================================
\section{Advanced Features}
% ===================================================================

\begin{frame}[fragile]{Code Listings}
    The `fragile` option is required for frames containing code listings.
    \begin{exampleblock}{Python Code Example}
\begin{lstlisting}[language=Python]
def greet(name):
    # This is a comment
    print(f"Hello, {name}!")

# Call the function
greet("ISRO")
\end{lstlisting}
    \end{exampleblock}
\end{frame}

\begin{frame}{Full-Frame Background Image}
    The `framebackground{}` command sets an image for a single slide.
    \framebackground[0.1]{images/color.png}
    \begin{alertblock}{Content on Top}
        Any content on the slide will appear on top of the background image. You can control the image opacity, e.g., `framebackground[0.1]{...}`.
    \end{alertblock}
\end{frame}

% ===================================================================
% --- Concluding Slides ---
% ===================================================================

\begin{frame}[allowframebreaks]{References}
    \printbibliography
\end{frame}

\begin{frame}[plain] 
    \vfill
    \centering
    \begin{beamercolorbox}[center, rounded=true, shadow=false]{block body}
        \Huge{\textbf{\textcolor{ISROBlue}{Thank You!}}}
    \end{beamercolorbox}
    \vfill
\end{frame}

\end{document}